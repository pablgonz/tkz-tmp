\section{Some examples}
\subsection{Some interesting examples}

\subsubsection{Similar isosceles triangles}

The following is from the excellent site \textbf{Descartes et les Mathématiques}. I did not modify the text and I am only the author of the programming of the figures.

\url{http://debart.pagesperso-orange.fr/seconde/triangle.html}

Bibliography:

\begin{itemize}

\item   Géométrie au Bac - Tangente, special issue no. 8 - Exercise 11, page 11


\item   Elisabeth Busser and Gilles Cohen: 200 nouveaux problèmes du "Monde" - POLE 2007 (200 new problems of "Le Monde")


\item   Affaire de logique n° 364 - Le Monde February 17, 2004
\end{itemize}


Two statements were proposed, one by the magazine \textit{Tangente} and the other by \textit{Le Monde}.

\vspace*{2cm}
\emph{Editor of the magazine "Tangente"}: \textcolor{orange}{Two similar isosceles triangles $AXB$ and $BYC$ are constructed with main vertices $X$ and $Y$, such that $A$, $B$ and $C$ are aligned and that these triangles are "indirect". Let $\alpha$ be the angle at vertex $\widehat{AXB}$ = $\widehat{BYC}$. We then construct a third isosceles triangle $XZY$ similar to the first two, with main vertex $Z$ and "indirect".
We ask to demonstrate that point $Z$ belongs to the straight line $(AC)$.}

\vspace*{2cm}
\emph{Editor of  "Le Monde"}: \textcolor{orange}{We construct two similar isosceles triangles $AXB$ and $BYC$ with principal vertices $X$ and $Y$, such that $A$, $B$ and $C$ are aligned and that these triangles are "indirect". Let $\alpha$ be the angle at vertex $\widehat{AXB}$ = $\widehat{BYC}$. The point Z of the line segment $[AC]$ is equidistant from the two vertices $X$ and $Y$.\\
At what angle does he see these two vertices?}

\vspace*{2cm} The constructions and their associated codes are on the next two pages, but you can search before looking. The programming respects (it seems to me ...) my reasoning in both cases.

 \subsubsection{Revised version of "Tangente"}
\begin{tkzexample}[]
\begin{tikzpicture}[scale=.8,rotate=60]
  \tkzDefPoint(6,0){X}   \tkzDefPoint(3,3){Y}
  \tkzDefShiftPoint[X](-110:6){A}    \tkzDefShiftPoint[X](-70:6){B}
  \tkzDefShiftPoint[Y](-110:4.2){A'} \tkzDefShiftPoint[Y](-70:4.2){B'}
  \tkzDefPointBy[translation= from A' to B ](Y) \tkzGetPoint{Y}
  \tkzDefPointBy[translation= from A' to B ](B') \tkzGetPoint{C}
  \tkzInterLL(A,B)(X,Y) \tkzGetPoint{O}
  \tkzDefMidPoint(X,Y) \tkzGetPoint{I}
  \tkzDefPointWith[orthogonal](I,Y)
  \tkzInterLL(I,tkzPointResult)(A,B) \tkzGetPoint{Z}
  \tkzDefCircle[circum](X,Y,B) \tkzGetPoint{O}
  \tkzDrawCircle(O,X)
  \tkzDrawLines[add = 0 and 1.5](A,C) \tkzDrawLines[add = 0 and 3](X,Y)
  \tkzDrawSegments(A,X B,X B,Y C,Y)   \tkzDrawSegments[color=red](X,Z Y,Z)
  \tkzDrawPoints(A,B,C,X,Y,O,Z)
  \tkzLabelPoints(A,B,C,Z)   \tkzLabelPoints[above right](X,Y,O)
\end{tikzpicture}
\end{tkzexample}

\subsubsection{"Le Monde" version}

\begin{tkzexample}[]
\begin{tikzpicture}[scale=1.25]
  \tkzDefPoint(0,0){A}
  \tkzDefPoint(3,0){B}
  \tkzDefPoint(9,0){C}
  \tkzDefPoint(1.5,2){X}
  \tkzDefPoint(6,4){Y}
  \tkzDefCircle[circum](X,Y,B) \tkzGetPoint{O}
  \tkzDefMidPoint(X,Y)               \tkzGetPoint{I}
  \tkzDefPointWith[orthogonal](I,Y)  \tkzGetPoint{i}
  \tkzDrawLines[add = 2 and 1,color=orange](I,i)
  \tkzInterLL(I,i)(A,B)              \tkzGetPoint{Z}
  \tkzInterLC(I,i)(O,B)              \tkzGetSecondPoint{M}
  \tkzDefPointWith[orthogonal](B,Z)  \tkzGetPoint{b}
  \tkzDrawCircle(O,B)
  \tkzDrawLines[add = 0 and 2,color=orange](B,b)
  \tkzDrawSegments(A,X B,X B,Y C,Y A,C X,Y)
  \tkzDrawSegments[color=red](X,Z Y,Z)
  \tkzDrawPoints(A,B,C,X,Y,Z,M,I)
  \tkzLabelPoints(A,B,C,Z)
  \tkzLabelPoints[above right](X,Y,M,I)
\end{tikzpicture}
\end{tkzexample}

\subsubsection{Triangle altitudes}

The following is again from the excellent site \textbf{Descartes et les Mathématiques} (Descartes and the Mathematics).

\url{http://debart.pagesperso-orange.fr/geoplan/geometrie_triangle.html}

The three altitudes of a triangle intersect at the same H-point.

\begin{tkzexample}[latex=7cm]
\begin{tikzpicture}[scale=.8]
   \tkzDefPoint(0,0){C}
   \tkzDefPoint(7,0){B}
   \tkzDefPoint(5,6){A}
   \tkzDrawPolygon(A,B,C)
   \tkzDefMidPoint(C,B)
   \tkzGetPoint{I}
   \tkzDrawArc(I,B)(C)
   \tkzInterLC(A,C)(I,B)
   \tkzGetSecondPoint{B'}
   \tkzInterLC(A,B)(I,B)
   \tkzGetFirstPoint{C'}
   \tkzInterLL(B,B')(C,C')
   \tkzGetPoint{H}
   \tkzInterLL(A,H)(C,B)
   \tkzGetPoint{A'}
     \tkzDefCircle[circum](A,B',C')
    \tkzGetPoint{O}
   \tkzDrawCircle[color=red](O,A)
   \tkzDrawSegments[color=orange](B,B' C,C' A,A')
   \tkzMarkRightAngles(C,B',B B,C',C C,A',A)
   \tkzDrawPoints(A,B,C,A',B',C',H)
   \tkzLabelPoints(A,B,C,A',B',C',H)
\end{tikzpicture}
\end{tkzexample}

\subsubsection{Altitudes - other construction}

\begin{tkzexample}[latex=7cm]
\begin{tikzpicture}[scale=.75]
  \tkzDefPoint(0,0){A}
  \tkzDefPoint(8,0){B}
  \tkzDefPoint(3.5,10){C}
  \tkzDefMidPoint(A,B)
  \tkzGetPoint{O}
  \tkzDefPointBy[projection=onto A--B](C)
  \tkzGetPoint{P}
  \tkzInterLC(C,A)(O,A)
  \tkzGetSecondPoint{M}
  \tkzInterLC(C,B)(O,A)
  \tkzGetFirstPoint{N}
  \tkzInterLL(B,M)(A,N)
  \tkzGetPoint{I}
  \tkzDrawCircle[diameter](A,B)
  \tkzDrawSegments(C,A C,B A,B B,M A,N)
  \tkzMarkRightAngles[fill=brown!20](A,M,B A,N,B A,P,C)
  \tkzDrawSegment[style=dashed,color=orange](C,P)
  \tkzLabelPoints(O,A,B,P)
  \tkzLabelPoint[left](M){$M$}
  \tkzLabelPoint[right](N){$N$}
  \tkzLabelPoint[above](C){$C$}
  \tkzLabelPoint[above right](I){$I$}
  \tkzDrawPoints[color=red](M,N,P,I)
  \tkzDrawPoints[color=brown](O,A,B,C)
\end{tikzpicture}
\end{tkzexample}

\subsection{Different authors}

\subsubsection{ Square root of the integers}
How to get $1$, $\sqrt{2}$, $\sqrt{3}$ with a rule and a compass.

\begin{tkzexample}[latex=7cm,small]
\begin{tikzpicture}[scale=1.5]
  \tkzDefPoint(0,0){O}
  \tkzDefPoint(1,0){a0}
   \tkzDrawSegment[blue](O,a0)
  \foreach \i [count=\j] in {0,...,10}{%
    \tkzDefPointWith[orthogonal normed](a\i,O)
    \tkzGetPoint{a\j}
    \tkzDrawPolySeg[color=blue](a\i,a\j,O)}
 \end{tikzpicture}
\end{tkzexample}


\subsubsection{About right triangle}

We have a segment $[AB]$ and we want to determine a point $C$ such that $AC=8$~cm    and $ABC$ is a right triangle in $B$.

\begin{tkzexample}[latex=7cm]
\begin{tikzpicture}[scale=.5]
  \tkzDefPoint["$A$" left](2,1){A}
  \tkzDefPoint(6,4){B}
  \tkzDrawSegment(A,B)
  \tkzDrawPoint[color=red](A)
  \tkzDrawPoint[color=red](B)
  \tkzDefPointWith[orthogonal,K=-1](B,A)
  \tkzDrawLine[add = .5 and .5](B,tkzPointResult)
  \tkzInterLC[R](B,tkzPointResult)(A,8 cm)
  \tkzGetPoints{C}{J}
  \tkzDrawPoint[color=red](C)
  \tkzCompass(A,C)
  \tkzMarkRightAngle(A,B,C)
  \tkzDrawLine[color=gray,style=dashed](A,C)
\end{tikzpicture}
\end{tkzexample}


\subsubsection{Archimedes}

This is an ancient problem   proved by the great Greek mathematician Archimedes .
The figure below shows a semicircle, with diameter $AB$. A tangent line is drawn and  touches the semicircle at $B$.   An other tangent line at a point, $C$, on the semicircle is drawn. We project the point $C$ on the line segment $[AB]$  on a point $D$. The two tangent lines intersect at the point $T$.

Prove that the line $(AT)$ bisects $(CD)$

\begin{tkzexample}[]
\begin{tikzpicture}[scale=1.25]
  \tkzDefPoint(0,0){A}\tkzDefPoint(6,0){D}
  \tkzDefPoint(8,0){B}\tkzDefPoint(4,0){I}
  \tkzDefLine[orthogonal=through D](A,D)
  \tkzInterLC[R](D,tkzPointResult)(I,4 cm) \tkzGetFirstPoint{C}
  \tkzDefLine[orthogonal=through C](I,C)    \tkzGetPoint{c}
  \tkzDefLine[orthogonal=through B](A,B)    \tkzGetPoint{b}
  \tkzInterLL(C,c)(B,b) \tkzGetPoint{T}
  \tkzInterLL(A,T)(C,D) \tkzGetPoint{P}
  \tkzDrawArc(I,B)(A)
  \tkzDrawSegments(A,B A,T C,D I,C) \tkzDrawSegment[color=orange](I,C)
  \tkzDrawLine[add = 1 and 0](C,T)   \tkzDrawLine[add = 0 and 1](B,T)
  \tkzMarkRightAngle(I,C,T)
  \tkzDrawPoints(A,B,I,D,C,T)
  \tkzLabelPoints(A,B,I,D)  \tkzLabelPoints[above right](C,T)
  \tkzMarkSegment[pos=.25,mark=s|](C,D) \tkzMarkSegment[pos=.75,mark=s|](C,D)
\end{tikzpicture}
\end{tkzexample}

\subsubsection{Example: Dimitris Kapeta}

You need in this example to use \tkzname{mkpos=.2} with \tkzcname{tkzMarkAngle} because the measure of $ \widehat{CAM}$ is too small.
Another possiblity is to use \tkzcname{tkzFillAngle}.


\begin{tkzexample}[]
\begin{tikzpicture}[scale=1.25]
  \tkzDefPoint(0,0){O}
  \tkzDefPoint(2.5,0){N}
  \tkzDefPoint(-4.2,0.5){M}
  \tkzDefPointBy[rotation=center O angle 30](N)
  \tkzGetPoint{B}
  \tkzDefPointBy[rotation=center O angle -50](N)
  \tkzGetPoint{A}
  \tkzInterLC(M,B)(O,N) \tkzGetFirstPoint{C}
  \tkzInterLC(M,A)(O,N) \tkzGetSecondPoint{A'}
  \tkzMarkAngle[mkpos=.2, size=0.5](A,C,B)
  \tkzMarkAngle[mkpos=.2, size=0.5](A,M,C)
  \tkzDrawSegments(A,C M,A M,B)
  \tkzDrawCircle(O,N)
  \tkzLabelCircle[above left](O,N)(120){$\mathcal{C}$}
  \tkzMarkAngle[mkpos=.2, size=1.2](C,A,M)
  \tkzDrawPoints(O, A, B, M, B, C)
  \tkzLabelPoints[right](O,A,B)
  \tkzLabelPoints[above left](M,C)
  \tkzLabelPoint[below left](A'){$A'$}
\end{tikzpicture}
\end{tkzexample}


\subsubsection{Example 1: John Kitzmiller }

Prove that $\bigtriangleup LKJ$ is equilateral.


\begin{tkzexample}[vbox,small]
\begin{tikzpicture}[scale=2]
  \tkzDefPoint[label=below left:A](0,0){A}
  \tkzDefPoint[label=below right:B](6,0){B}
  \tkzDefTriangle[equilateral](A,B) \tkzGetPoint{C}
  \tkzMarkSegments[mark=|](A,B A,C B,C)
  \tkzDefBarycentricPoint(A=1,B=2) \tkzGetPoint{C'}
  \tkzDefBarycentricPoint(A=2,C=1) \tkzGetPoint{B'}
  \tkzDefBarycentricPoint(C=2,B=1) \tkzGetPoint{A'}
  \tkzInterLL(A,A')(C,C') \tkzGetPoint{J}
  \tkzInterLL(C,C')(B,B') \tkzGetPoint{K}
  \tkzInterLL(B,B')(A,A') \tkzGetPoint{L}
  \tkzLabelPoint[above](C){C}
  \tkzDrawPolygon(A,B,C) \tkzDrawSegments(A,J B,L C,K)
  \tkzMarkAngles[size=1 cm](J,A,C K,C,B L,B,A)
  \tkzMarkAngles[thick,size=1 cm](A,C,J C,B,K B,A,L)
  \tkzMarkAngles[opacity=.5](A,C,J C,B,K B,A,L)
  \tkzFillAngles[fill= orange,size=1 cm,opacity=.3](J,A,C K,C,B L,B,A)
  \tkzFillAngles[fill=orange, opacity=.3,thick,size=1,](A,C,J C,B,K B,A,L)
  \tkzFillAngles[fill=green, size=1, opacity=.5](A,C,J C,B,K B,A,L)
  \tkzFillPolygon[color=yellow, opacity=.2](J,A,C)
  \tkzFillPolygon[color=yellow, opacity=.2](K,B,C)
  \tkzFillPolygon[color=yellow, opacity=.2](L,A,B)
  \tkzDrawSegments[line width=3pt,color=cyan,opacity=0.4](A,J C,K B,L)
  \tkzDrawSegments[line width=3pt,color=red,opacity=0.4](A,L B,K C,J)
  \tkzMarkSegments[mark=o](J,K K,L L,J)
  \tkzLabelPoint[right](J){J}
  \tkzLabelPoint[below](K){K}
  \tkzLabelPoint[above left](L){L}
\end{tikzpicture}
\end{tkzexample}

\subsubsection{Example 2:  John Kitzmiller }
Prove that $\dfrac{AC}{CE}=\dfrac{BD}{DF}$.

Another interesting example from John, you can see how to use some extra options like \tkzname{decoration} and \tkzname{postaction}  from \TIKZ\ with \tkzname{tkz-euclide}.

\begin{tkzexample}[vbox,small]
\begin{tikzpicture}[scale=2,decoration={markings,
  mark=at position 3cm with {\arrow[scale=2]{>}}}]
  \tkzDefPoints{0/0/E, 6/0/F, 0/1.8/P, 6/1.8/Q, 0/3/R, 6/3/S}
  \tkzDrawLines[postaction={decorate}](E,F P,Q R,S)
  \tkzDefPoints{3.5/3/A, 5/3/B}
  \tkzDrawSegments(E,A F,B)
  \tkzInterLL(E,A)(P,Q) \tkzGetPoint{C}
  \tkzInterLL(B,F)(P,Q) \tkzGetPoint{D}
  \tkzLabelPoints[above right](A,B)
  \tkzLabelPoints[below](E,F)
  \tkzLabelPoints[above left](C)
  \tkzDrawSegments[style=dashed](A,F)
  \tkzInterLL(A,F)(P,Q) \tkzGetPoint{G}
  \tkzLabelPoints[above right](D,G)
  \tkzDrawSegments[color=teal, line width=3pt, opacity=0.4](A,C A,G)
  \tkzDrawSegments[color=magenta, line width=3pt, opacity=0.4](C,E G,F)
  \tkzDrawSegments[color=teal, line width=3pt, opacity=0.4](B,D)
  \tkzDrawSegments[color=magenta, line width=3pt, opacity=0.4](D,F)
\end{tikzpicture}
\end{tkzexample}

\subsubsection{Example 3:  John Kitzmiller }
Prove that $\dfrac{BC}{CD}=\dfrac{AB}{AD} \qquad$ (Angle Bisector).

\begin{tkzexample}[vbox,small]
\begin{tikzpicture}[scale=2]
  \tkzDefPoints{0/0/B, 5/0/D}       \tkzDefPoint(70:3){A}
  \tkzDrawPolygon(B,D,A)
  \tkzDefLine[bisector](B,A,D)      \tkzGetPoint{a}
  \tkzInterLL(A,a)(B,D)           \tkzGetPoint{C}
  \tkzDefLine[parallel=through B](A,C) \tkzGetPoint{b}
  \tkzInterLL(A,D)(B,b)           \tkzGetPoint{P}
  \begin{scope}[decoration={markings,
   mark=at position .5 with {\arrow[scale=2]{>}}}]
   \tkzDrawSegments[postaction={decorate},dashed](C,A P,B)
  \end{scope}
  \tkzDrawSegment(A,C) \tkzDrawSegment[style=dashed](A,P)
  \tkzLabelPoints[below](B,C,D) \tkzLabelPoints[above](A,P)
  \tkzDrawSegments[color=magenta, line width=3pt, opacity=0.4](B,C P,A)
  \tkzDrawSegments[color=teal,    line width=3pt, opacity=0.4](C,D A,D)
  \tkzDrawSegments[color=magenta, line width=3pt, opacity=0.4](A,B)
  \tkzMarkAngles[size=3mm](B,A,C C,A,D)
  \tkzMarkAngles[size=3mm](B,A,C A,B,P)
  \tkzMarkAngles[size=3mm](B,P,A C,A,D)
  \tkzMarkAngles[size=3mm](B,A,C A,B,P B,P,A C,A,D)
  \tkzFillAngles[fill=green,  opacity=0.5](B,A,C A,B,P)
  \tkzFillAngles[fill=yellow, opacity=0.3](B,P,A C,A,D)
  \tkzFillAngles[fill=green,  opacity=0.6](B,A,C A,B,P B,P,A C,A,D)
  \tkzLabelAngle[pos=1](B,A,C){1}   \tkzLabelAngle[pos=1](C,A,D){2}
  \tkzLabelAngle[pos=1](A,B,P){3}    \tkzLabelAngle[pos=1](B,P,A){4}
  \tkzMarkSegments[mark=|](A,B A,P)
\end{tikzpicture}
\end{tkzexample}


\subsubsection{Example 4: author John Kitzmiller }
Prove that $\overline{AG}\cong\overline{EF} \qquad$ (Detour).

\begin{tkzexample}[vbox,small]
\begin{tikzpicture}[scale=2]
  \tkzDefPoint(0,3){A}    \tkzDefPoint(6,3){E}  \tkzDefPoint(1.35,3){B}
  \tkzDefPoint(4.65,3){D} \tkzDefPoint(1,1){G}  \tkzDefPoint(5,5){F}
  \tkzDefMidPoint(A,E)    \tkzGetPoint{C}
  \tkzFillPolygon[yellow, opacity=0.4](B,G,C)
  \tkzFillPolygon[yellow, opacity=0.4](D,F,C)
  \tkzFillPolygon[blue, opacity=0.3](A,B,G)
  \tkzFillPolygon[blue, opacity=0.3](E,D,F)
  \tkzMarkAngles[size=0.5 cm](B,G,A D,F,E)
  \tkzMarkAngles[size=0.5 cm](B,C,G D,C,F)
  \tkzMarkAngles[size=0.5 cm](G,B,C F,D,C)
  \tkzMarkAngles[size=0.5 cm](A,B,G E,D,F)
  \tkzFillAngles[size=0.5 cm,fill=green](B,G,A D,F,E)
  \tkzFillAngles[size=0.5 cm,fill=orange](B,C,G D,C,F)
  \tkzFillAngles[size=0.5 cm,fill=yellow](G,B,C F,D,C)
  \tkzFillAngles[size=0.5 cm,fill=red](A,B,G E,D,F)
  \tkzMarkSegments[mark=|](B,C D,C)  \tkzMarkSegments[mark=s||](G,C F,C)
  \tkzMarkSegments[mark=o](A,G E,F)  \tkzMarkSegments[mark=s](B,G D,F)
  \tkzDrawSegment[color=red](A,E)
  \tkzDrawSegment[color=blue](F,G)
  \tkzDrawSegments(A,G G,B E,F F,D)
  \tkzLabelPoints[below](C,D,E,G)  \tkzLabelPoints[above](A,B,F)
\end{tikzpicture}
\end{tkzexample}

\subsubsection{Example 1: from Indonesia}

\begin{tkzexample}[vbox,small]
\begin{tikzpicture}[scale=3]
   \tkzDefPoints{0/0/A,2/0/B}
   \tkzDefSquare(A,B) \tkzGetPoints{C}{D}
   \tkzDefPointBy[rotation=center D angle 45](C)\tkzGetPoint{G}
   \tkzDefSquare(G,D)\tkzGetPoints{E}{F}
   \tkzInterLL(B,C)(E,F)\tkzGetPoint{H}
   \tkzFillPolygon[gray!10](D,E,H,C,D)
   \tkzDrawPolygon(A,...,D)\tkzDrawPolygon(D,...,G)
   \tkzDrawSegment(B,E)
   \tkzMarkSegments[mark=|,size=3pt,color=gray](A,B B,C C,D D,A E,F F,G G,D D,E)
   \tkzMarkSegments[mark=||,size=3pt,color=gray](B,E E,H)
   \tkzLabelPoints[left](A,D)
   \tkzLabelPoints[right](B,C,F,H)
   \tkzLabelPoints[above](G)\tkzLabelPoints[below](E)
   \tkzMarkRightAngles(D,A,B D,G,F)
\end{tikzpicture}
\end{tkzexample}

\subsubsection{Example 2: from Indonesia}
\begin{tkzexample}[vbox,small]
  \begin{tikzpicture}[pol/.style={fill=brown!40,opacity=.5},
                     seg/.style={tkzdotted,color=gray},
                     hidden pt/.style={fill=gray!40},
                     mra/.style={color=gray!70,tkzdotted,/tkzrightangle/size=.2},
                     scale=3]
  \tkzSetUpPoint[size=2]                
  \tkzDefPoints{0/0/A,2.5/0/B,1.33/0.75/D,0/2.5/E,2.5/2.5/F}
  \tkzDefLine[parallel=through D](A,B) \tkzGetPoint{I1}
  \tkzDefLine[parallel=through B](A,D) \tkzGetPoint{I2}
  \tkzInterLL(D,I1)(B,I2) \tkzGetPoint{C}
  \tkzDefLine[parallel=through E](A,D) \tkzGetPoint{I3}
  \tkzDefLine[parallel=through D](A,E) \tkzGetPoint{I4}
  \tkzInterLL(E,I3)(D,I4) \tkzGetPoint{H}
  \tkzDefLine[parallel=through F](E,H) \tkzGetPoint{I5}
  \tkzDefLine[parallel=through H](E,F) \tkzGetPoint{I6}
  \tkzInterLL(F,I5)(H,I6) \tkzGetPoint{G}
  \tkzDefMidPoint(G,H) \tkzGetPoint{P}
  \tkzDefMidPoint(G,C) \tkzGetPoint{Q}
  \tkzDefMidPoint(B,C) \tkzGetPoint{R}
  \tkzDefMidPoint(A,B) \tkzGetPoint{S}
  \tkzDefMidPoint(A,E) \tkzGetPoint{T}
  \tkzDefMidPoint(E,H) \tkzGetPoint{U}
  \tkzDefMidPoint(A,D) \tkzGetPoint{M}
  \tkzDefMidPoint(D,C) \tkzGetPoint{N}
  \tkzInterLL(B,D)(S,R) \tkzGetPoint{L}
  \tkzInterLL(H,F)(U,P) \tkzGetPoint{K}
  \tkzDefLine[parallel=through K](D,H) \tkzGetPoint{I7}
  \tkzInterLL(K,I7)(B,D) \tkzGetPoint{O}
  
  \tkzFillPolygon[pol](P,Q,R,S,T,U)
  \tkzDrawSegments[seg](K,O K,L P,Q R,S T,U 
                    C,D H,D A,D M,N B,D)
  \tkzDrawSegments(E,H B,C G,F G,H G,C Q,R S,T U,P H,F)
  \tkzDrawPolygon(A,B,F,E)
  \tkzDrawPoints(A,B,C,E,F,G,H,P,Q,R,S,T,U,K)
  \tkzDrawPoints[hidden pt](M,N,O,D)
  \tkzMarkRightAngle[mra](L,O,K)
  \tkzMarkSegments[mark=|,size=1pt,thick,color=gray](A,S B,S B,R C,R 
                    Q,C Q,G G,P H,P 
                    E,U H,U E,T A,T)
  
  \tkzLabelAngle[pos=.3](K,L,O){$\alpha$}
  \tkzLabelPoints[below](O,A,S,B)
  \tkzLabelPoints[above](H,P,G)
  \tkzLabelPoints[left](T,E)
  \tkzLabelPoints[right](C,Q)
  \tkzLabelPoints[above left](U,D,M)
  \tkzLabelPoints[above right](L,N)
  \tkzLabelPoints[below right](F,R)
  \tkzLabelPoints[below left](K)
  \end{tikzpicture}
\end{tkzexample}


\subsubsection{Three circles}

\begin{tkzexample}[vbox,small]
\begin{tikzpicture}[scale=1.5]
  \tkzDefPoints{0/0/A,8/0/B,0/4/a,8/4/b,8/8/c}
  \tkzDefTriangle[equilateral](A,B) \tkzGetPoint{C}
  \tkzDrawPolygon(A,B,C)
  \tkzDefSquare(A,B) \tkzGetPoints{D}{E}
  \tkzClipBB
  \tkzDefMidPoint(A,B) \tkzGetPoint{M}
  \tkzDefMidPoint(B,C) \tkzGetPoint{N}
  \tkzDefMidPoint(A,C) \tkzGetPoint{P}
  \tkzDrawSemiCircle[gray,dashed](M,B)
  \tkzDrawSemiCircle[gray,dashed](A,M)
  \tkzDrawSemiCircle[gray,dashed](A,B)
  \tkzDrawCircle[gray,dashed](B,A)
  \tkzInterLL(A,N)(M,a) \tkzGetPoint{Ia}
  \tkzDefPointBy[projection = onto A--B](Ia)
  \tkzGetPoint{ha}
  \tkzDrawCircle[gray](Ia,ha)
  \tkzInterLL(B,P)(M,b) \tkzGetPoint{Ib}
  \tkzDefPointBy[projection = onto A--B](Ib)
  \tkzGetPoint{hb}
  \tkzDrawCircle[gray](Ib,hb)
  \tkzInterLL(A,c)(M,C) \tkzGetPoint{Ic}
  \tkzDefPointBy[projection = onto A--C](Ic)
  \tkzGetPoint{hc}
  \tkzDrawCircle[gray](Ic,hc)
  \tkzInterLL(A,Ia)(B,Ib) \tkzGetPoint{G}
  \tkzDrawCircle[gray,dashed](G,Ia)
  \tkzDrawPolySeg(A,E,D,B)
  \tkzDrawPoints(A,B,C)
  \tkzDrawPoints(G,Ia,Ib,Ic)
  \tkzDrawSegments[gray,dashed](C,M A,N B,P M,a M,b A,a a,b b,B A,D Ia,ha)
\end{tikzpicture}
\end{tkzexample}

\subsubsection{"The" Circle of APOLLONIUS}

\begin{tkzexample}[vbox,small]
  \begin{tikzpicture}[scale=.5]
  \tkzDefPoints{0/0/A,6/0/B,0.8/4/C}
  \tkzDefTriangleCenter[euler](A,B,C)        \tkzGetPoint{N} 
  \tkzDefTriangleCenter[circum](A,B,C)       \tkzGetPoint{O} 
  \tkzDefTriangleCenter[lemoine](A,B,C)      \tkzGetPoint{K} 
  \tkzDefTriangleCenter[spieker](A,B,C)      \tkzGetPoint{Sp}
  \tkzDefExCircle(A,B,C)     \tkzGetPoint{Jb}
  \tkzDefExCircle(C,A,B)     \tkzGetPoint{Ja}
  \tkzDefExCircle(B,C,A)     \tkzGetPoint{Jc}
  \tkzDefPointBy[projection=onto B--C ](Jc)   \tkzGetPoint{Xc}
  \tkzDefPointBy[projection=onto B--C ](Jb)   \tkzGetPoint{Xb}
  \tkzDefPointBy[projection=onto A--B ](Ja)   \tkzGetPoint{Za}
  \tkzDefPointBy[projection=onto A--B ](Jb)   \tkzGetPoint{Zb}
  \tkzDefLine[parallel=through Xc](A,C)       \tkzGetPoint{X'c}
  \tkzDefLine[parallel=through Xb](A,B)       \tkzGetPoint{X'b}
  \tkzDefLine[parallel=through Za](C,A)       \tkzGetPoint{Z'a}
  \tkzDefLine[parallel=through Zb](C,B)       \tkzGetPoint{Z'b}
  \tkzInterLL(Xc,X'c)(A,B)                    \tkzGetPoint{B'}
  \tkzInterLL(Xb,X'b)(A,C)                    \tkzGetPoint{C'}
  \tkzInterLL(Za,Z'a)(C,B)                    \tkzGetPoint{A''}
  \tkzInterLL(Zb,Z'b)(C,A)                    \tkzGetPoint{B''}
  \tkzDefPointBy[reflection= over Jc--Jb](B') \tkzGetPoint{Ca}
  \tkzDefPointBy[reflection= over Jc--Jb](C') \tkzGetPoint{Ba}
  \tkzDefPointBy[reflection= over Ja--Jb](A'')\tkzGetPoint{Bc}
  \tkzDefPointBy[reflection= over Ja--Jb](B'')\tkzGetPoint{Ac}
  \tkzDefCircle[circum](Ac,Ca,Ba)             \tkzGetPoint{Q}
  \tkzDrawCircle[circum](Ac,Ca,Ba)
  \tkzDefPointWith[linear,K=1.1](Q,Ac)        \tkzGetPoint{nAc}
  \tkzClipCircle[through](Q,nAc)
  \tkzDrawLines[add=1.5 and 1.5,dashed](A,B B,C A,C)
  \tkzDrawPolygon[color=blue](A,B,C)
  \tkzDrawPolygon[dashed,color=blue](Ja,Jb,Jc)
  \tkzDrawCircles[ex](A,B,C B,C,A C,A,B) 
  \tkzDrawLines[add=0 and 0,dashed](Ca,Bc B,Za A,Ba B',C')
  \tkzDrawLine[add=1 and 1,dashed](Xb,Xc)
  \tkzDrawLine[add=7 and 3,blue](O,K)
  \tkzDrawLine[add=8 and 15,red](N,Sp)
  \tkzDrawLines[add=10 and 10](K,O N,Sp)
  \tkzDrawSegments(Ba,Ca Bc,Ac)
  \tkzDrawPoints(A,B,C,N,Ja,Jb,Jc,Xb,Xc,B',C',Za,Zb,Ba,Ca,Bc,Ac,Q,Sp,K,O)
  \tkzLabelPoints(A,B,C,N,Ja,Jb,Jc,Xb,Xc,B',C',Za,Zb,Ba,Ca,Bc,Ac,Q,Sp)
  \tkzLabelPoints[above](K,O)
  \end{tikzpicture}
\end{tkzexample}


 
\endinput
