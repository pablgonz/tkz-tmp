\section{Defining points using a vector}

\subsection{\tkzcname{tkzDefPointWith}}
There are several possibilities to create points that meet certain vector conditions.
This can be done with \tkzcname{tkzDefPointWith}. The general principle is as follows, two points are passed as arguments, i.e. a vector. The different options allow to obtain a new point forming with the first point (with some exceptions) a collinear vector or a vector orthogonal to the first vector. Then the length is either proportional to that of the first one, or proportional to the unit. Since this point is only used temporarily, it does not have to be named immediately. The result is in \tkzname{tkzPointResult}. The macro \tkzNameMacro{tkzGetPoint} allows you to retrieve the point and name it differently.

 There are options to define the distance between the given point and the obtained point.
In the general case this distance is the distance between the 2 points given as arguments if the option is of the "normed" type then the distance between the given point and the obtained point is 1 cm. Then the $K$ option allows to obtain multiples.

\begin{NewMacroBox}{tkzDefPointWith}{\parg{pt1,pt2}}%
 It is in fact the definition of a point meeting vectorial conditions.

\medskip

\begin{tabular}{lll}%
\toprule
arguments             & definition & explication                         \\
\midrule
\TAline{(pt1,pt2)} {point couple}{the result is a point in \tkzname{tkzPointResult} } \\

\bottomrule
\end{tabular}

\medskip
In what follows, it is assumed that the point is recovered by \tkzNameMacro{tkzGetPoint\{C\}}

\begin{tabular}{lll}%
\toprule
options             & example & explication                         \\
\midrule
\TOline{orthogonal}{[orthogonal](A,B)}{$AC=AB$ and $\overrightarrow{AC} \perp \overrightarrow{AB}$}
\TOline{orthogonal normed}{[orthogonal normed](A,B)}{$AC=1$ and $\overrightarrow{AC} \perp \overrightarrow{AB}$}
\TOline{linear}{[linear](A,B)}{$\overrightarrow{AC}=K \times \overrightarrow{AB}$}
\TOline{linear normed}{[linear normed](A,B)}{$AC=K$ and $\overrightarrow{AC}=k\times \overrightarrow{AB}$ }
\TOline{colinear= at \#1}{[colinear= at C](A,B)}{$\overrightarrow{CD}= \overrightarrow{AB}$ }
\TOline{colinear normed= at \#1}{[colinear normed= at C](A,B)}{$\overrightarrow{CD}= \overrightarrow{AB}$ }
\TOline{K}{[linear](A,B),K=2}{$\overrightarrow{AC}=2\times \overrightarrow{AB}$}
\end{tabular}
\end{NewMacroBox}

\subsubsection{Option \tkzname{colinear at}}
 $(\overrightarrow{AB}=\overrightarrow{CD})$
\begin{tkzexample}[latex=6cm,small]
\begin{tikzpicture}[scale=1.2,
   vect/.style={->,shorten >=3pt,>=latex'}]
  \tkzDefPoint(2,3){A}   \tkzDefPoint(4,2){B}
  \tkzDefPoint(0,1){C}
  \tkzDefPointWith[colinear=at C](A,B)
  \tkzGetPoint{D}
  \tkzDrawPoints[color=red](A,B,C,D)
  \tkzLabelPoints[above right=3pt](A,B,C,D)
  \tkzDrawSegments[vect](A,B C,D)
\end{tikzpicture}
\end{tkzexample}


\subsubsection{Option \tkzname{colinear at} with $K$}

\begin{tkzexample}[latex=7cm,small]
\begin{tikzpicture}[vect/.style={->,
                 shorten >=3pt,>=latex'}]
  \tkzDefPoint(0,0){A}
  \tkzDefPoint(5,0){B}
  \tkzDefPoint(1,2){C}
  \tkzDefPointWith[colinear=at C](A,B)
  \tkzGetPoint{G}
  \tkzDefPointWith[colinear=at C,K=0.5](A,B)
  \tkzGetPoint{H}
  \tkzLabelPoints(A,B,C,G,H)
  \tkzDrawPoints(A,B,C,G,H)
    \tkzDrawSegments[vect](A,B C,H)
\end{tikzpicture}
\end{tkzexample}

\subsubsection{Option \tkzname{colinear at} with $K=\frac{\sqrt{2}}{2}$}
\begin{tkzexample}[latex=6cm,small]
\begin{tikzpicture}[vect/.style={->,
              shorten >=3pt,>=latex'}]
 \tkzDefPoint(1,1){A}
 \tkzDefPoint(4,2){B}
 \tkzDefPoint(2,2){CU}
 \tkzDefPointWith[colinear=at C,K=sqrt(2)/2](A,B)
 \tkzGetPoint{D}
 \tkzDrawPoints[color=red](A,B,C,D)
 \tkzDrawSegments[vect](A,B C,D)
\end{tikzpicture}
\end{tkzexample}

\subsubsection{Option \tkzname{orthogonal}}
AB=AC since $K=1$.
\begin{tkzexample}[latex=6cm,small]
\begin{tikzpicture}[scale=1.2,
  vect/.style={->,shorten >=3pt,>=latex'}]
   \tkzDefPoint(2,3){A}
       \tkzDefPoint(4,2){B}
   \tkzDefPointWith[orthogonal,K=1](A,B)
     \tkzGetPoint{C}
   \tkzDrawPoints[color=red](A,B,C)
   \tkzLabelPoints[right=3pt](B,C)
   \tkzLabelPoints[below=3pt](A)
   \tkzDrawSegments[vect](A,B A,C)
   \tkzMarkRightAngle(B,A,C)
\end{tikzpicture}
\end{tkzexample}



\subsubsection{Option \tkzname{orthogonal} with $K=-1$}
OK=OI since $\lvert K \rvert=1$ then OI=OJ=OK.

\begin{tkzexample}[latex=7cm,small]
\begin{tikzpicture}[scale=.75]
  \tkzDefPoint(1,2){O}
  \tkzDefPoint(2,5){I}
  \tkzDefPointWith[orthogonal](O,I)
  \tkzGetPoint{J}
  \tkzDefPointWith[orthogonal,K=-1](O,I)
  \tkzGetPoint{K}
  \tkzDrawSegment(O,I)
  \tkzDrawSegments[->](O,J O,K)
  \tkzMarkRightAngles(I,O,J I,O,K)
  \tkzDrawPoints(O,I,J,K)
  \tkzLabelPoints(O,I,J,K)
\end{tikzpicture}
\end{tkzexample}

\subsubsection{Option \tkzname{orthogonal} more complicated example}
\begin{tkzexample}[latex=7cm,small]
\begin{tikzpicture}[scale=.75]
  \tkzDefPoints{0/0/A,6/0/B}
  \tkzDefMidPoint(A,B)
      \tkzGetPoint{I}
  \tkzDefPointWith[orthogonal,K=-.75](B,A)
  \tkzGetPoint{C}
  \tkzInterLC(B,C)(B,I)
       \tkzGetPoints{D}{F}
  \tkzDuplicateSegment(B,F)(A,F)
  \tkzGetPoint{E}
  \tkzDrawArc[delta=10](F,E)(B)
  \tkzInterLC(A,B)(A,E)
      \tkzGetPoints{N}{M}
  \tkzDrawArc[delta=10](A,M)(E)
  \tkzDrawLines(A,B B,C A,F)
  \tkzCompass(B,F)
  \tkzDrawPoints(A,B,C,F,M,E)
  \tkzLabelPoints(A,B,C,F,M,E)
\end{tikzpicture}
\end{tkzexample}

\subsubsection{Options \tkzname{colinear} and \tkzname{orthogonal}}
\begin{tkzexample}[latex=7cm,small]
\begin{tikzpicture}[scale=1.2,
  vect/.style={->,shorten >=3pt,>=latex'}]
   \tkzDefPoint(2,1){A}
   \tkzDefPoint(6,2){B}
   \tkzDefPointWith[orthogonal,K=.5](A,B)
   \tkzGetPoint{C}
   \tkzDefPointWith[colinear=at C,K=.5](A,B)
   \tkzGetPoint{D}
   \tkzMarkRightAngle[fill=gray!20](B,A,C)
   \tkzDrawSegments[vect](A,B A,C C,D)
   \tkzDrawPoints(A,...,D)
\end{tikzpicture}
\end{tkzexample}

\subsubsection{Option  \tkzname{orthogonal normed}, $K=1$}
$AC=1$.

\begin{tkzexample}[latex=7cm,small]
\begin{tikzpicture}[scale=1.2,
  vect/.style={->,shorten >=3pt,>=latex'}]
  \tkzDefPoint(2,3){A}   \tkzDefPoint(4,2){B}
  \tkzDefPointWith[orthogonal normed](A,B)
  \tkzGetPoint{C}
  \tkzDrawPoints[color=red](A,B,C)
  \tkzDrawSegments[vect](A,B A,C)
  \tkzMarkRightAngle[fill=gray!20](B,A,C)
\end{tikzpicture}
\end{tkzexample}

\subsubsection{Option \tkzname{orthogonal normed} and $K=2$}
$K=2$ therefore $AC=2$.

\begin{tkzexample}[latex=7cm,small]
\begin{tikzpicture}[scale=1.2,
   vect/.style={->,shorten >=3pt,>=latex'}]
   \tkzDefPoint(2,3){A}   \tkzDefPoint(5,1){B}
   \tkzDefPointWith[orthogonal normed,K=2](A,B)
   \tkzGetPoint{C}
   \tkzDrawPoints[color=red](A,B,C)
   \tkzDrawCircle[R](A,2cm)
   \tkzDrawSegments[vect](A,B A,C)
   \tkzMarkRightAngle[fill=gray!20](B,A,C)
   \tkzLabelPoints[above=3pt](A,B,C)
\end{tikzpicture}
\end{tkzexample}

\subsubsection{Option \tkzname{linear}}
Here $K=0.5$.

This amounts to applying a homothety or a multiplication of a vector by a real. Here is the middle of $[AB]$.

\begin{tkzexample}[latex=7cm,small]
\begin{tikzpicture}[scale=1.2]
   \tkzDefPoint(1,3){A}   \tkzDefPoint(4,2){B}
   \tkzDefPointWith[linear,K=0.5](A,B)
   \tkzGetPoint{C}
   \tkzDrawPoints[color=red](A,B,C)
   \tkzDrawSegment(A,B)
   \tkzLabelPoints[above right=3pt](A,B,C)
\end{tikzpicture}
\end{tkzexample}

\subsubsection{Option \tkzname{linear normed}}
In the following example $AC=1$ and $C$ belongs to $(AB)$.

\begin{tkzexample}[latex=7cm,small]
\begin{tikzpicture}[scale=1.2]
 \tkzDefPoint(1,3){A}   \tkzDefPoint(4,2){B}
 \tkzDefPointWith[linear normed](A,B)
 \tkzGetPoint{C}
 \tkzDrawPoints[color=red](A,B,C)
 \tkzDrawSegment(A,B)
 \tkzLabelSegment(A,C){$1$}
 \tkzLabelPoints[above right=3pt](A,B,C)
\end{tikzpicture}
\end{tkzexample}



%<--------------------------------------------------------------------------–>
%         tkzGetVectxy
%<--------------------------------------------------------------------------–>


\subsection{\tkzcname{tkzGetVectxy} }
Retrieving the coordinates of a vector.

\begin{NewMacroBox}{tkzGetVectxy}{\parg{$A,B$}\var{text}}%
Allows to obtain the coordinates of a vector.

\medskip
\begin{tabular}{lll}%
\toprule
arguments    & example & explication      \\

\midrule

\TAline{(point)\{name of macro\}} {\tkzcname{tkzGetVectxy}(A,B)\{V\}}{\tkzcname{Vx},\tkzcname{Vy}: coordinates of $\overrightarrow{AB}$}
\end{tabular}
\end{NewMacroBox}

\subsubsection{Coordinate transfer with \tkzcname{tkzGetVectxy}}

\begin{tkzexample}[latex=7cm,small]
\begin{tikzpicture}
 \tkzDefPoint(0,0){O}
 \tkzDefPoint(1,1){A}
 \tkzDefPoint(4,2){B}
 \tkzGetVectxy(A,B){v}
 \tkzDefPoint(\vx,\vy){V}
 \tkzDrawSegment[->,color=red](O,V)
 \tkzDrawSegment[->,color=blue](A,B)
 \tkzDrawPoints(A,B,O)
 \tkzLabelPoints(A,B,O,V)
\end{tikzpicture}
\end{tkzexample}



\endinput
