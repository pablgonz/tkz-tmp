\section{Definition of polygons}
\subsection{Defining the points of a square} \label{def_square}
We have seen the definitions of some triangles. Let us look at the definitions of some quadrilaterals and regular polygons.

\begin{NewMacroBox}{tkzDefSquare}{\parg{pt1,pt2}}%
The square is defined in the forward direction. From two points, two more points are obtained such that the four taken in order form a square. The square is defined in the forward direction.    The results are in \tkzname{tkzFirstPointResult} and \tkzname{tkzSecondPointResult}.\\
We can rename them with \tkzcname{tkzGetPoints}.

\medskip
\begin{tabular}{lll}%
\toprule
Arguments             & example & explication                         \\
\midrule
\TAline{\parg{pt1,pt2}}{\tkzcname{tkzDefSquare}\parg{A,B}}{The square is defined in the direct direction.}
\end{tabular}
\end{NewMacroBox}

\subsubsection{Using \tkzcname{tkzDefSquare} with two points}
Note the inversion of the first two points and the result.

\begin{tkzexample}[latex=4cm,small]
\begin{tikzpicture}[scale=.5]
  \tkzDefPoint(0,0){A} \tkzDefPoint(3,0){B}
  \tkzDefSquare(A,B)
  \tkzDrawPolygon[color=red](A,B,tkzFirstPointResult,%
               tkzSecondPointResult)
  \tkzDefSquare(B,A)
  \tkzDrawPolygon[color=blue](B,A,tkzFirstPointResult,%
               tkzSecondPointResult)
\end{tikzpicture}
\end{tkzexample}

 We may only need one point to draw an isosceles right-angled triangle so we use \tkzcname{tkzGetFirstPoint} or \tkzcname{tkzGetSecondPoint}.

\subsubsection{Use of \tkzcname{tkzDefSquare} to obtain an isosceles right-angled triangle}
\begin{tkzexample}[latex=7cm,small]
\begin{tikzpicture}[scale=1]
  \tkzDefPoint(0,0){A}
  \tkzDefPoint(3,0){B}
  \tkzDefSquare(A,B) \tkzGetFirstPoint{C}
  \tkzDrawPolygon[color=blue,fill=blue!30](A,B,C)
\end{tikzpicture}
\end{tkzexample}

\subsubsection{Pythagorean Theorem and \tkzcname{tkzDefSquare} }
\begin{tkzexample}[latex=8cm,small]
\begin{tikzpicture}[scale=.5]
\tkzInit
\tkzDefPoint(0,0){C}
\tkzDefPoint(4,0){A}
\tkzDefPoint(0,3){B}
\tkzDefSquare(B,A)\tkzGetPoints{E}{F}
\tkzDefSquare(A,C)\tkzGetPoints{G}{H}
\tkzDefSquare(C,B)\tkzGetPoints{I}{J}
\tkzFillPolygon[fill = red!50 ](A,C,G,H)
\tkzFillPolygon[fill = blue!50 ](C,B,I,J)
\tkzFillPolygon[fill = purple!50](B,A,E,F)
\tkzFillPolygon[fill = orange,opacity=.5](A,B,C)
\tkzDrawPolygon[line width = 1pt](A,B,C)
\tkzDrawPolygon[line width = 1pt](A,C,G,H)
\tkzDrawPolygon[line width = 1pt](C,B,I,J)
\tkzDrawPolygon[line width = 1pt](B,A,E,F)
\tkzLabelSegment[](A,C){$a$}
\tkzLabelSegment[](C,B){$b$}
\tkzLabelSegment[swap](A,B){$c$}
\end{tikzpicture}
\end{tkzexample}

\subsection{Definition of parallelogram}

\subsection{Defining the points of a parallelogram}
It is a matter of completing three points in order to obtain a parallelogram.
\begin{NewMacroBox}{tkzDefParallelogram}{\parg{pt1,pt2,pt3}}%
From three points, another point is obtained such that the four taken in order form a parallelogram.  The result is in \tkzname{tkzPointResult}. \\
We can rename it with the name \tkzcname{tkzGetPoint}...

\begin{tabular}{lll}%
\toprule
arguments &  default & definition  \\
\midrule
\TAline{\parg{pt1,pt2,pt3}}{no default}{Three points are necessary}
\bottomrule
\end{tabular}
\end{NewMacroBox}

\subsubsection{Example of a parallelogram definition}

\begin{tkzexample}[latex=7 cm,small]
\begin{tikzpicture}[scale=1]
 \tkzDefPoints{0/0/A,3/0/B,4/2/C}
 \tkzDefParallelogram(A,B,C)
 \tkzGetPoint{D}
 \tkzDrawPolygon(A,B,C,D)
 \tkzLabelPoints(A,B)
 \tkzLabelPoints[above right](C,D)
 \tkzDrawPoints(A,...,D)
\end{tikzpicture}
\end{tkzexample}



\subsubsection{Simple example}
Explanation of the definition of a parallelogram
\begin{tkzexample}[latex=7 cm,small]
\begin{tikzpicture}[scale=1]
  \tkzDefPoints{0/0/A,3/0/B,4/2/C}
  \tkzDefPointWith[colinear= at C](B,A)
  \tkzGetPoint{D}
  \tkzDrawPolygon(A,B,C,D)
  \tkzLabelPoints(A,B)
  \tkzLabelPoints[above right](C,D)
  \tkzDrawPoints(A,...,D)
\end{tikzpicture}
\end{tkzexample}

\subsubsection{Construction of the golden rectangle }

\begin{tkzexample}[latex=8cm,small]
\begin{tikzpicture}[scale=.5]
  \tkzInit[xmax=14,ymax=10]
  \tkzClip[space=1]
  \tkzDefPoint(0,0){A}
  \tkzDefPoint(8,0){B}
  \tkzDefMidPoint(A,B)\tkzGetPoint{I}
  \tkzDefSquare(A,B)\tkzGetPoints{C}{D}
  \tkzDrawSquare(A,B)
  \tkzInterLC(A,B)(I,C)\tkzGetPoints{G}{E}
  \tkzDrawArc[style=dashed,color=gray](I,E)(D)
  \tkzDefPointWith[colinear= at C](E,B)
  \tkzGetPoint{F}
  \tkzDrawPoints(C,D,E,F)
  \tkzLabelPoints(A,B,C,D,E,F)
  \tkzDrawSegments[style=dashed,color=gray]%
(E,F C,F B,E)
\end{tikzpicture}
\end{tkzexample}




\subsection{Drawing a square}
\begin{NewMacroBox}{tkzDrawSquare}{\oarg{local options}\parg{pt1,pt2}}%
The macro draws a square but not the vertices. It is possible to color the inside. The order of the points is that of the direct direction of the trigonometric circle.

\medskip
\begin{tabular}{lll}%
\toprule
arguments             & example & explication                         \\
\midrule
\TAline{\parg{pt1,pt2}}{|\tkzcname{tkzDrawSquare}|\parg{A,B}}{|\tkzcname{tkzGetPoints\{C\}\{D\}}|}
\bottomrule
\end{tabular}

\medskip
\begin{tabular}{lll}%
options             & example & explication                         \\
\midrule
\TOline{Options TikZ}{|red,line width=1pt|}{}
\end{tabular}
\end{NewMacroBox}

\subsubsection{The idea is to inscribe two squares in a semi-circle.}

\begin{tkzexample}[latex=6 cm,small]
\begin{tikzpicture}[scale=.75]
   \tkzInit[ymax=8,xmax=8]
 \tkzClip[space=.25]    \tkzDefPoint(0,0){A}
 \tkzDefPoint(8,0){B}  \tkzDefPoint(4,0){I}
 \tkzDefSquare(A,B)    \tkzGetPoints{C}{D}
 \tkzInterLC(I,C)(I,B) \tkzGetPoints{E'}{E}
 \tkzInterLC(I,D)(I,B) \tkzGetPoints{F'}{F}
 \tkzDefPointsBy[projection=onto A--B](E,F){H,G}
 \tkzDefPointsBy[symmetry   = center H](I){J}
 \tkzDefSquare(H,J)    \tkzGetPoints{K}{L}
 \tkzDrawSector[fill=yellow](I,B)(A)
 \tkzFillPolygon[color=red!40](H,E,F,G)
 \tkzFillPolygon[color=blue!40](H,J,K,L)
 \tkzDrawPolySeg[color=red](H,E,F,G)
 \tkzDrawPolySeg[color=red](J,K,L)
 \tkzDrawPoints(E,G,H,F,J,K,L)
\end{tikzpicture}
\end{tkzexample}

\subsection{The golden rectangle}
 \begin{NewMacroBox}{tkzDefGoldRectangle}{\parg{point,point}}%
The macro determines a rectangle whose size ratio is the number $\Phi$. The created points are in \tkzname{tkzFirstPointResult} and \tkzname{tkzSecondPointResult}. They can be obtained with the macro \tkzcname{tkzGetPoints}. The following macro is used to draw the rectangle.

\begin{tabular}{lll}%
\toprule
arguments             & example & explication                         \\
\midrule
\TAline{\parg{pt1,pt2}}{\parg{A,B}}{If C and D are created then $AB/BC=\Phi$.}
 \end{tabular}
\end{NewMacroBox}

 \begin{NewMacroBox}{tkzDrawGoldRectangle}{\oarg{local options}\parg{point,point}}
\begin{tabular}{lll}%
arguments             & example & explication                         \\
\midrule
\TAline{\parg{pt1,pt2}}{\parg{A,B}}{Draws the golden rectangle based on the segment $[AB]$}
\end{tabular}

\medskip
\begin{tabular}{lll}%
options     & example & explication     \\
\midrule
\TOline{Options TikZ}{|red,line width=1pt|}{}
\end{tabular}
\end{NewMacroBox}

\subsubsection{Golden Rectangles}
\begin{tkzexample}[latex=6 cm,small]
\begin{tikzpicture}[scale=.6]
 \tkzDefPoint(0,0){A}      \tkzDefPoint(8,0){B}
 \tkzDefGoldRectangle(A,B) \tkzGetPoints{C}{D}
 \tkzDefGoldRectangle(B,C) \tkzGetPoints{E}{F}
 \tkzDrawPolygon[color=red,fill=red!20](A,B,C,D)
 \tkzDrawPolygon[color=blue,fill=blue!20](B,C,E,F)
\end{tikzpicture}
\end{tkzexample}

\subsection{Drawing a polygon}
 \begin{NewMacroBox}{tkzDrawPolygon}{\oarg{local options}\parg{points list}}%
Just give a list of points and the macro plots the polygon using the \TIKZ\ options present. You can  replace $(A,B,C,D,E)$ by $(A,...,E)$ and $(P_1,P_2,P_3,P_4,P_5)$ by $(P_1,P...,P_5)$

\begin{tabular}{lll}%
\toprule
arguments             & example & explication                         \\
\midrule
\TAline{\parg{pt1,pt2,pt3,...}}{|\BS tkzDrawPolygon[gray,dashed](A,B,C)|}{Drawing a triangle}
 \end{tabular}

\medskip
\begin{tabular}{lll}%
\toprule
options             & default & example                         \\
\midrule
\TOline{Options TikZ}{...}{|\BS tkzDrawPolygon[red,line width=2pt](A,B,C)|}
 \end{tabular}
\end{NewMacroBox}

\subsubsection{\tkzcname{tkzDrawPolygon}}

\begin{tkzexample}[latex=7cm, small]
\begin{tikzpicture} [rotate=18,scale=1.5]
 \tkzDefPoint(0,0){A}
 \tkzDefPoint(2.25,0.2){B}
 \tkzDefPoint(2.5,2.75){C}
 \tkzDefPoint(-0.75,2){D}
 \tkzDrawPolygon[fill=black!50!blue!20!](A,B,C,D)
 \tkzDrawSegments[style=dashed](A,C B,D)
\end{tikzpicture}\end{tkzexample}

\subsection{Drawing a polygonal chain}
 \begin{NewMacroBox}{tkzDrawPolySeg}{\oarg{local options}\parg{points list}}%
Just give a list of points and the macro plots the polygonal chain using the \TIKZ\ options present.

\begin{tabular}{lll}%
\toprule
arguments             & example & explication                         \\
\midrule
\TAline{\parg{pt1,pt2,pt3,...}}{|\BS tkzDrawPolySeg[gray,dashed](A,B,C)|}{Drawing a triangle}
 \end{tabular}

\medskip
\begin{tabular}{lll}%
\toprule
options             & default & example                         \\
\midrule
\TOline{Options TikZ}{...}{|\BS tkzDrawPolySeg[red,line width=2pt](A,B,C)|}
 \end{tabular}
\end{NewMacroBox}

\subsubsection{Polygonal chain}

\begin{tkzexample}[latex=7cm, small]
\begin{tikzpicture}
 \tkzDefPoints{0/0/A,6/0/B,3/4/C,2/2/D}
 \tkzDrawPolySeg(A,...,D)
 \tkzDrawPoints(A,...,D)
\end{tikzpicture}
\end{tkzexample}

\subsubsection{Polygonal chain: index notation}

\begin{tkzexample}[latex=7cm, small]
\begin{tikzpicture}
\foreach \pt in {1,2,...,8} {%
\tkzDefPoint(\pt*20:3){P_\pt}}
\tkzDrawPolySeg(P_1,P_...,P_8)
\tkzDrawPoints(P_1,P_...,P_8)
\end{tikzpicture}
\end{tkzexample}

\subsection{Clip a polygon}
 \begin{NewMacroBox}{tkzClipPolygon}{\oarg{local options}\parg{points list}}%
This macro makes it possible to contain the different plots in the designated polygon.

\medskip
\begin{tabular}{lll}%
\toprule
arguments       & example & explication     \\
\midrule
\TAline{\parg{pt1,pt2}}{\parg{A,B}}{}
%\bottomrule
 \end{tabular}
\end{NewMacroBox}

\subsubsection{\tkzcname{tkzClipPolygon}}
\begin{tkzexample}[latex=7 cm,small]
\begin{tikzpicture}[scale=1.25]
 \tkzInit[xmin=0,xmax=4,ymin=0,ymax=3]
 \tkzClip[space=.5]
 \tkzDefPoint(0,0){A} \tkzDefPoint(4,0){B}
 \tkzDefPoint(1,3){C} \tkzDrawPolygon(A,B,C)
 \tkzDefPoint(0,2){D}  \tkzDefPoint(2,0){E}
 \tkzDrawPoints(D,E) \tkzLabelPoints(D,E)
 \tkzClipPolygon(A,B,C)
 \tkzDrawLine[color=red](D,E)
\end{tikzpicture}
\end{tkzexample}

\subsubsection{Example: use of "Clip" for Sangaku in a square}
\begin{tkzexample}[latex=7cm, small]
\begin{tikzpicture}[scale=.75]
 \tkzDefPoint(0,0){A} \tkzDefPoint(8,0){B}
 \tkzDefSquare(A,B) \tkzGetPoints{C}{D}
 \tkzDrawPolygon(B,C,D,A)
 \tkzClipPolygon(B,C,D,A)
 \tkzDefPoint(4,8){F}
 \tkzDefTriangle[equilateral](C,D)
 \tkzGetPoint{I}
 \tkzDrawPoint(I)
 \tkzDefPointBy[projection=onto B--C](I)
 \tkzGetPoint{J}
 \tkzInterLL(D,B)(I,J)  \tkzGetPoint{K}
 \tkzDefPointBy[symmetry=center K](B)
 \tkzGetPoint{M}
 \tkzDrawCircle(M,I)
 \tkzCalcLength(M,I)   \tkzGetLength{dMI}
 \tkzFillPolygon[color = orange](A,B,C,D)
 \tkzFillCircle[R,color = yellow](M,\dMI pt)
 \tkzFillCircle[R,color = blue!50!black](F,4 cm)%
\end{tikzpicture}
\end{tkzexample}

\subsection{Color a polygon}
 \begin{NewMacroBox}{tkzFillPolygon}{\oarg{local options}\parg{points list}}%
You can color by drawing the polygon, but in this case you color the inside of the polygon without drawing it.

\medskip
\begin{tabular}{lll}%
\toprule
arguments                & example & explication                         \\
\midrule
\TAline{\parg{pt1,pt2,\dots}}{\parg{A,B,\dots}}{}
%\bottomrule
 \end{tabular}
\end{NewMacroBox}

\subsubsection{\tkzcname{tkzFillPolygon}}
\begin{tkzexample}[latex=7cm, small]
\begin{tikzpicture}[scale=0.7]
\tkzInit[xmin=-3,xmax=6,ymin=-1,ymax=6]
\tkzDrawX[noticks]
\tkzDrawY[noticks]
\tkzDefPoint(0,0){O}  \tkzDefPoint(4,2){A}
\tkzDefPoint(-2,6){B}
\tkzPointShowCoord[xlabel=$x$,ylabel=$y$](A)
\tkzPointShowCoord[xlabel=$x'$,ylabel=$y'$,%
                   ystyle={right=2pt}](B)
\tkzDrawSegments[->](O,A O,B)
\tkzLabelSegment[above=3pt](O,A){$\vec{u}$}
\tkzLabelSegment[above=3pt](O,B){$\vec{v}$}
\tkzMarkAngle[fill= yellow,size=1.8cm,%
              opacity=.5](A,O,B)
\tkzFillPolygon[red!30,opacity=0.25](A,B,O)
\tkzLabelAngle[pos = 1.5](A,O,B){$\alpha$}
\end{tikzpicture}
\end{tkzexample}

\subsection{Regular polygon}
 \begin{NewMacroBox}{tkzDefRegPolygon}{\oarg{local options}\parg{pt1,pt2}}%
From the number of sides, depending on the options, this macro determines a regular polygon according to its center or one side.

\begin{tabular}{lll}%
\toprule
arguments             & example & explication                         \\
\midrule
\TAline{\parg{pt1,pt2}}{\parg{O,A}}{with option "center", $O$ is the center of the polygon.}
\TAline{\parg{pt1,pt2}}{\parg{A,B}}{with option "side", $[AB]$ is a side.}
 \end{tabular}

\medskip
\begin{tabular}{lll}%
\toprule
options             & default & example                         \\
\midrule
\TOline{name}{P}{The vertices are named $P1$,$P2$,\dots}
\TOline{sides}{5}{number of sides.}
\TOline{center}{center}{The first point is the center.}
\TOline{side}{center}{The two points are vertices.}
\TOline{Options TikZ}{...}{}
\end{tabular}
\end{NewMacroBox}

\subsubsection{Option \tkzname{center}}
\begin{tkzexample}[latex=7cm, small]
\begin{tikzpicture}
    \tkzDefPoints{0/0/P0,0/0/Q0,2/0/P1}
    \tkzDefMidPoint(P0,P1) \tkzGetPoint{Q1}
  \tkzDefRegPolygon[center,sides=7](P0,P1)
    \tkzDefMidPoint(P1,P2) \tkzGetPoint{Q1}
  \tkzDefRegPolygon[center,sides=7,name=Q](P0,Q1)
    \tkzDrawPolygon(P1,P...,P7)
    \tkzFillPolygon[gray!20](Q0,Q1,P2,Q2)
    \foreach \j in {1,...,7} {\tkzDrawSegment[black](P0,Q\j)}
\end{tikzpicture}
\end{tkzexample}

\subsubsection{Option \tkzname{side}}
\begin{tkzexample}[latex=7cm, small]
\begin{tikzpicture}[scale=1]
    \tkzDefPoints{-4/0/A, -1/0/B}
    \tkzDefRegPolygon[side,sides=5,name=P](A,B)
    \tkzDrawPolygon[thick](P1,P...,P5)
\end{tikzpicture}
\end{tkzexample}
\endinput
