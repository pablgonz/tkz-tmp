% tkz-obj-eu-points.tex
% Copyright 2020  Alain Matthes
% This work may be distributed and/or modified under the
% conditions of the LaTeX Project Public License, either version 1.3
% of this license or (at your option) any later version.
% The latest version of this license is in
%   http://www.latex-project.org/lppl.txt
% and version 1.3 or later is part of all distributions of LaTeX
% version 2005/12/01 or later.
%
% This work has the LPPL maintenance status “maintained”.
%
% The Current Maintainer of this work is Alain Matthes.
%  utf8 encoding
\def\fileversion{3.05c}
\def\filedate{2020/03/08}
\typeout{2020/03/08 3.05c  tkz-obj-eu-points.tex}
\makeatletter
%add ExCenter
%<--------------------------------------------------------------------------–>
%                   Specific points
%<--------------------------------------------------------------------------–>
%                              barycentre
%<--------------------------------------------------------------------------–>
\def\tkzDefBarycentricPoint(#1){%
\begingroup
 \path[coordinate]  (barycentric cs:#1) coordinate (tkzPointResult);
\endgroup
}
\let\tkzDefBCPoint\tkzDefBarycentricPoint

\def\tkzDefCentroid(#1){%
\begingroup
   \xdef\tkz@arg{}
	 \foreach \X in {#1}
        {%
		  \ifx\tkzutil@empty\tkz@arg%
		   \xdef\tkz@arg{\X=1}
        \else
         \xdef\tkz@arg{\tkz@arg,\X=1}
        \fi
		  }
   \path[coordinate] (barycentric cs:\tkz@arg) coordinate (tkzPointResult);
\endgroup
}

%<--------------------------------------------------------------------------–>
%                      milieu  de deux points
%<--------------------------------------------------------------------------–>
% possible   \coordinate (#3) at ($(#1)!0.5!(#2)$);
%<--------------------------------------------------------------------------–>
% \def\tkzDefMidPoint(#1,#2){%
% \begingroup
%  \path (#1) -- (#2) coordinate[pos=.5](tkzPointResult);
% \endgroup
% }
\def\tkzDefMidPoint(#1,#2){%
\begingroup
   \pgf@process{\pgfpointanchor{#1}{center}}%
   \tkz@ax\pgf@x%
   \tkz@ay\pgf@y%
   \pgf@process{\pgfpointanchor{#2}{center}}%
   \tkz@bx\pgf@x%
   \tkz@by\pgf@y%
   \advance\tkz@bx by\tkz@ax\relax%
   \advance\tkz@by by\tkz@ay\relax%
   \divide\tkz@bx by2\relax%
   \divide\tkz@by by2\relax
   \pgfcoordinate{tkzPointResult}{\pgfqpoint{\tkz@bx}{\tkz@by}}
\endgroup
}
\def\tkz@DefMidPoint(#1,#2,#3,#4){%
\begingroup
   \tkz@ax#1%
   \tkz@ay#2%
   \tkz@bx#3%
   \tkz@by#4%
   \advance\tkz@bx by\tkz@ax\relax%
   \advance\tkz@by by\tkz@ay\relax%
   \divide\tkz@bx by2\relax%
   \divide\tkz@by by2\relax
   \pgfcoordinate{tkzPointResult}{\pgfqpoint{\tkz@bx}{\tkz@by}}
\endgroup
}
%<--------------------------------------------------------------------------–>
%                    Internal Similitude center
%  Two circles have two similitude centers namely the internal center of similitude Si and the external similitude center Se.
%<--------------------------------------------------------------------------–>
\def\tkzDefIntSimilitudeCenter(#1,#2)(#3,#4){%
\begingroup
   \path[coordinate](barycentric cs:#1=#4,#3=#2)coordinate (tkzPointResult);
\endgroup
}
\let\tkzIntSimilitudeCenter\tkzDefIntSimilitudeCenter
%<--------------------------------------------------------------------------–>
%                    External Similitude center
%<--------------------------------------------------------------------------–>
\def\tkzDefExtSimilitudeCenter(#1,#2)(#3,#4){%
\begingroup
   \path[coordinate](barycentric cs:#1=-#4,#3=#2) coordinate(tkzPointResult);
\endgroup
}
\let\tkzExtSimilitudeCenter\tkzDefExtSimilitudeCenter
%<--------------------------------------------------------------------------–>
%                    triangle center
%<--------------------------------------------------------------------------–>

\def\tkz@numtc{0}
\pgfkeys{/tkzDefTriangleCenter/.cd,
   ortho/.code        = \def\tkz@numtc{0},
   centroid/.code     = \def\tkz@numtc{1},
   median/.code       = \def\tkz@numtc{1},
   circum/.code       = \def\tkz@numtc{2},
   in/.code           = \def\tkz@numtc{3},
   ex/.code           = \def\tkz@numtc{4},
   euler/.code        = \def\tkz@numtc{5},
   symmedian/.code    = \def\tkz@numtc{6},
   lemoine/.code      = \def\tkz@numtc{6},
   grebe/.code        = \def\tkz@numtc{6},
   spieker/.code      = \def\tkz@numtc{7},
   gergonne/.code     = \def\tkz@numtc{8},
   nagel/.code        = \def\tkz@numtc{9},
   mittenpunkt/.code  = \def\tkz@numtc{10},
   feuerbach/.code    = \def\tkz@numtc{11},
   circum
}
\def\tkzDefTriangleCenter{\pgfutil@ifnextchar[{\tkz@DefTriangleCenter}{\tkz@DefTriangleCenter[]}}
\def\tkz@DefTriangleCenter[#1](#2){%
\begingroup
\pgfqkeys{/tkzDefTriangleCenter}{#1}
\ifcase\tkz@numtc%
  \tkzOrthoCenter(#2)
  \or% 1
  \tkzCentroid(#2)
  \or% 2
  \tkzCircumCenter(#2)
  \or% 3
  \tkzInCenter(#2)
  \or% 4
  \tkzExCenter(#2)
  \or% 5
  \tkzEulerCenter(#2)
  \or% 6
  \tkzSymmedianCenter(#2)
  \or% 7
  \tkzSpiekerCenter(#2)
  \or% 8
  \tkzGergonneCenter(#2)
  \or%9
  \tkzNagelCenter(#2)
  \or%10
  \tkzMittenpunktCenter(#2)
  \or%11
  \tkzFeuerbachCenter(#2)
    \fi
\endgroup
}
%<--------------------------------------------------------------------------–>
%                    OrthoCenter
%<--------------------------------------------------------------------------–>
\def\tkzOrthoCenter(#1,#2,#3){%  H orthocentre
\begingroup
  \pgfinterruptboundingbox
   \tkzUProjection(#1,#2)(#3)
   \pgfnodealias{ort@pta}{tkzPointResult}
   \tkzUProjection(#1,#3)(#2)
   \pgfnodealias{ort@ptb}{tkzPointResult}
   \tkzInterLL(#2,ort@ptb)(#3,ort@pta)
   \endpgfinterruptboundingbox
\endgroup
}
\let\tkzDefOrthoCenter\tkzOrthoCenter
%<--------------------------------------------------------------------------–>
%                      GravityCenter modif 3.03
%<--------------------------------------------------------------------------–>
\def\tkzCentroid(#1,#2,#3){%
\begingroup
   \pgf@process{\pgfpointanchor{#1}{center}}%
   \tkz@ax\pgf@x%
   \tkz@ay\pgf@y%
   \pgf@process{\pgfpointanchor{#2}{center}}%
   \tkz@bx\pgf@x%
   \tkz@by\pgf@y%
   \pgf@process{\pgfpointanchor{#3}{center}}%
   \tkz@cx\pgf@x%
   \tkz@cy\pgf@y%
   \advance\tkz@cx by\tkz@ax\relax%
   \advance\tkz@cy by\tkz@ay\relax%
   \advance\tkz@cx by\tkz@bx\relax%
   \advance\tkz@cy by\tkz@by\relax%
   \divide\tkz@cx by3\relax%
   \divide\tkz@cy by3\relax
   \pgfinterruptboundingbox
   \pgfcoordinate{tkzPointResult}{\pgfqpoint{\tkz@cx}{\tkz@cy}}
   \endpgfinterruptboundingbox
\endgroup
}
\let\tkzBaryCenter\tkzCentroid

%<--------------------------------------------------------------------------–>
%                      CircumCenter
%<--------------------------------------------------------------------------–>
\def\tkzCircumCenter(#1,#2,#3){%
\begingroup
\pgfinterruptboundingbox
 \tkzDefMediatorLine(#1,#2)
 \pgf@process{\pgfpointanchor{tkzFirstPointResult}{center}}%
 \tkz@ax\pgf@x%
 \tkz@ay\pgf@y%
 \pgf@process{\pgfpointanchor{tkzSecondPointResult}{center}}%
 \tkz@bx\pgf@x%
 \tkz@by\pgf@y%
 \tkzDefMediatorLine(#1,#3)
 \pgf@process{\pgfpointanchor{tkzFirstPointResult}{center}}%
 \tkz@cx\pgf@x%
 \tkz@cy\pgf@y%
 \pgf@process{\pgfpointanchor{tkzSecondPointResult}{center}}%
 \tkz@dx\pgf@x%
 \tkz@dy\pgf@y%
 \tkzInterLLxy(\tkz@ax,\tkz@ay,\tkz@bx,\tkz@by)(\tkz@cx,\tkz@cy,\tkz@dx,\tkz@dy)%
\endpgfinterruptboundingbox
\endgroup
}
\let\tkzDefCircumCenter\tkzCircumCenter
%<--------------------------------------------------------------------------–>
%                     InCenter
%<--------------------------------------------------------------------------–>
\def\tkzInCenter(#1,#2,#3){%
\begingroup
\pgfinterruptboundingbox
   \tkzDefBisectorLine(#3,#1,#2)
   \pgf@process{\pgfpointanchor{tkzPointResult}{center}}%
   \tkz@bx\pgf@x%
   \tkz@by\pgf@y%
   \tkzDefBisectorLine(#3,#2,#1)
   \pgf@process{\pgfpointanchor{tkzPointResult}{center}}%
   \tkz@dx\pgf@x%
   \tkz@dy\pgf@y%
   \pgf@process{\pgfpointanchor{#1}{center}}%
   \tkz@ax\pgf@x%
   \tkz@ay\pgf@y%
   \pgf@process{\pgfpointanchor{#2}{center}}%
   \tkz@cx\pgf@x%
   \tkz@cy\pgf@y%
   \tkzInterLLxy(\tkz@ax,\tkz@ay,\tkz@bx,\tkz@by)%
               (\tkz@cx,\tkz@cy,\tkz@dx,\tkz@dy)%
	\endpgfinterruptboundingbox
\endgroup
}
\let\tkzDefInCenter\tkzInCenter
%<--------------------------------------------------------------------------–>
%                     ExCenter
%<--------------------------------------------------------------------------–>
\def\tkzExCenter(#1,#2,#3){%
\begingroup
\pgfinterruptboundingbox
   \tkzDefBisectorOutLine(#2,#1,#3)
   \pgf@process{\pgfpointanchor{tkzPointResult}{center}}%
   \tkz@bx\pgf@x%
   \tkz@by\pgf@y%
    \tkzDefBisectorOutLine(#2,#3,#1)
   \pgf@process{\pgfpointanchor{tkzPointResult}{center}}%
   \tkz@dx\pgf@x%
   \tkz@dy\pgf@y%
   \pgf@process{\pgfpointanchor{#1}{center}}%
   \tkz@ax\pgf@x%
   \tkz@ay\pgf@y%
   \pgf@process{\pgfpointanchor{#3}{center}}%
   \tkz@cx\pgf@x%
   \tkz@cy\pgf@y%
   \tkzInterLLxy(\tkz@ax,\tkz@ay,\tkz@bx,\tkz@by)%
             (\tkz@cx,\tkz@cy,\tkz@dx,\tkz@dy)%
	\endpgfinterruptboundingbox
\endgroup
}
\let\tkzDefExCenter\tkzExCenter
%<--------------------------------------------------------------------------–>
%                     EulerCenter neuf points
%<--------------------------------------------------------------------------–>
\def\tkzEulerCenter(#1,#2,#3){%
% mileu de orthocentre et centre cercle circonscrit
% passe par les midpoints par les pieds des hauteurs
\begingroup
\pgfinterruptboundingbox
   \tkzDefMidPoint(#1,#2)
   \pgfnodealias{eu@mic}{tkzPointResult}
   \tkzDefMidPoint(#1,#3)
   \pgfnodealias{eu@mib}{tkzPointResult}
   \tkzDefMidPoint(#2,#3)
   \pgfnodealias{eu@mia}{tkzPointResult}
   \tkzCircumCenter(eu@mia,eu@mib,eu@mic)
\endpgfinterruptboundingbox
\endgroup
}
\let\tkzNinePointCenter\tkzEulerCenter
\let\tkzDefEulerCenter\tkzEulerCenter
%<--------------------------------------------------------------------------–>
%Symmedian center Lemoine point Grebe point K
%<--------------------------------------------------------------------------–>
\def\tkzSymmedianCenter(#1,#2,#3){%
\begingroup
\pgfinterruptboundingbox
  \tkzDefMidPoint(#2,#3)
  \pgfnodealias{eu@mic}{tkzPointResult}
  \tkzDefMidPoint(#1,#3)
  \pgfnodealias{eu@mib}{tkzPointResult}
  \tkzUProjection(#2,#3)(#1)
  \pgfnodealias{ort@pta}{tkzPointResult}
	\tkzDefMidPoint(#1,ort@pta)
  \pgfnodealias{eu@mid}{tkzPointResult}
  \tkzUProjection(#1,#3)(#2)
  \pgfnodealias{ort@ptb}{tkzPointResult}
	\tkzDefMidPoint(#2,ort@ptb)
  \pgfnodealias{eu@mie}{tkzPointResult}
  \tkzInterLL(eu@mic,eu@mid)(eu@mib,eu@mie)
\endpgfinterruptboundingbox
\endgroup
}
\let\tkzLemoinePoint\tkzSymmedianCenter
\let\tkzGrebePoint\tkzSymmedianCenter
\let\tkzDefLemoinePoint\tkzLemoinePoint
%<--------------------------------------------------------------------------–>
%                   Spieker center
%<--------------------------------------------------------------------------–>
\def\tkzSpiekerCenter(#1,#2,#3){%
\begingroup
% we need to get the midpoints
\pgfcoordinate{tkz@m3}{%
    \pgfpointscale{0.5}{%
   \pgfpointadd{\pgfpointanchor{#1}{center}}%
               {\pgfpointanchor{#2}{center}}}}%
\pgfcoordinate{tkz@m2}{%
    \pgfpointscale{0.5}{%
   \pgfpointadd{\pgfpointanchor{#1}{center}}%
               {\pgfpointanchor{#3}{center}}}}%
\pgfcoordinate{tkz@m1}{%
   \pgfpointscale{0.5}{%
   \pgfpointadd{\pgfpointanchor{#2}{center}}%
               {\pgfpointanchor{#3}{center}}}}%
\tkzInCenter(tkz@m1,tkz@m2,tkz@m3)
\endgroup
}
\let\tkzDefSpiekerCenter\tkzSpiekerCenter
%<--------------------------------------------------------------------------–>
%                    Gergonne center Ge
%<--------------------------------------------------------------------------–>
\def\tkzGergonneCenter(#1,#2,#3){%
\begingroup
\pgfinterruptboundingbox
   \tkzInCenter(#1,#2,#3)
   \pgfnodealias{tkz@ptin}{tkzPointResult}
   \tkzUProjection(#2,#3)(tkz@ptin)
   \pgfnodealias{tkz@oca}{tkzPointResult}
   \tkzUProjection(#1,#3)(tkz@ptin)
   \pgfnodealias{tkz@ocb}{tkzPointResult}
   \tkzInterLL(#1,tkz@oca)(#2,tkz@ocb)
\endpgfinterruptboundingbox
\endgroup
}
\let\tkzDefGergonneCenter\tkzGergonneCenter
%<--------------------------------------------------------------------------–>
%                    Nagel center Na
%<--------------------------------------------------------------------------–>
%  INa = 3 IG. Nagel point % correction 02/02/20
\def\tkzNagelCenter(#1,#2,#3){%
\begingroup
\pgfinterruptboundingbox
  \tkzDefExcentralTriangle(#1,#2,#3){tkz@a,tkz@b,tkz@c}
  \tkzUProjection(#2,#3)(tkz@a)
  \pgfnodealias{tkz@tgta}{tkzPointResult}
  \tkzUProjection(#1,#2)(tkz@c)
  \pgfnodealias{tkz@tgtc}{tkzPointResult}
  \tkzInterLL(#1,tkz@tgta)(#3,tkz@tgtc)
\endpgfinterruptboundingbox
\endgroup
}
\let\tkzDefNagelCenter\tkzNagelCenter
%<--------------------------------------------------------------------------–>
%  Mittenpunkt
%<--------------------------------------------------------------------------–>
\def\tkzMittenpunktCenter(#1,#2,#3){%
\begingroup
\pgfinterruptboundingbox
 \tkzExCenter(#2,#3,#1)
 \pgfnodealias{tkz@a}{tkzPointResult}
 \tkzExCenter(#3,#1,#2)
 \pgfnodealias{tkz@b}{tkzPointResult}
 \pgfcoordinate{tkz@ma}{%
 \pgfpointscale{0.5}{%
 \pgfpointadd{\pgfpointanchor{#1}{center}}{\pgfpointanchor{#2}{center}}}}%
 \pgfcoordinate{tkz@mb}{%
 \pgfpointscale{0.5}{%
 \pgfpointadd{\pgfpointanchor{#2}{center}}{\pgfpointanchor{#3}{center}}}}%
 \tkzInterLL(tkz@a,tkz@ma)(tkz@b,tkz@mb)
 \endpgfinterruptboundingbox
\endgroup
}
\let\tkzDefMittenpunktCenter\tkzMittenpunktCenter
\let\tkzDefMiddlespoint\tkzMittenpunktCenter
%<--------------------------------------------------------------------------–>
%                   Feuerbach point
%<--------------------------------------------------------------------------–>
\def\tkzFeuerbachCenter(#1,#2,#3){%
\begingroup
\pgfinterruptboundingbox
 \tkzEulerCenter(#1,#2,#3)
 \pgfnodealias{tkz@euler}{tkzPointResult}
 \tkzInCenter(#1,#2,#3)
 \pgfnodealias{tkz@in}{tkzPointResult}
 \tkzUProjection(#2,#3)(tkzPointResult)
 \tkzInterLC(tkz@in,tkz@euler)(tkz@in,tkzPointResult)\tkzGetFirstPoint{tkz@fe}
 \tkzRenamePoint(tkz@fe){tkzPointResult}
 \endpgfinterruptboundingbox
\endgroup
}
\let\tkzDefFeuerbachCenter\tkzFeuerbachCenter
%<--------------------------------------------------------------------------–>
%                     Orthogonal center
%<--------------------------------------------------------------------------–>
\def\tkzOrthogonalCenter(#1,#2){%
\begingroup
\pgfinterruptboundingbox
 \tkz@VecK[\tkz@koeff/(1+\tkz@koeff)](#1,#2)
 \pgfnodealias{tkzFirstPointResult}{tkzPointResult}
 \tkz@VecK[\tkz@koeff/(\tkz@koeff-1)](#1,#2)
 \pgfnodealias{tkzSecondPointResult}{tkzPointResult}
   \tkzDefMidPoint(tkzFirstPointResult,tkzSecondPointResult)
\endpgfinterruptboundingbox
\endgroup
}
%<--------------------------------------------------------------------------–>
%                  End Triangle center
%<--------------------------------------------------------------------------–>
%<--------------------------------------------------------------------------–>
%                  Projection  center of excircles
%<--------------------------------------------------------------------------–>
\def\tkzDefProjExcenter{\pgfutil@ifnextchar[{%
    \tkz@DefProjExcenter}{%
    \tkz@DefProjExcenter[]}
    }
\def\tkz@DefProjExcenter[#1](#2,#3,#4)(#5)#6{
\begingroup
  \SetUpPTTR{#1}
   \foreach \name  [count=\i] in {#5} {%
      \global\expandafter\edef\csname tkz@pt\i\endcsname{\name}
        }
   \foreach \name  [count=\i] in {#6} {%
      \global\expandafter\edef\csname tkz@ppt\i\endcsname{\name}
        }
\tkzDefPointBy[projection=onto #3--#4 ](\tkz@pttr@name \csname tkz@pt1\endcsname)
\pgfnodealias{\csname tkz@ppt1\endcsname\csname tkz@pt1\endcsname}{tkzPointResult}
\tkzDefPointBy[projection=onto #3--#4 ](\tkz@pttr@name \csname tkz@pt2\endcsname)
\pgfnodealias{\csname tkz@ppt1\endcsname\csname tkz@pt2\endcsname}{tkzPointResult}
\tkzDefPointBy[projection=onto #3--#4 ](\tkz@pttr@name \csname tkz@pt3\endcsname)
\pgfnodealias{\csname tkz@ppt1\endcsname\csname tkz@pt3\endcsname}{tkzPointResult}
\tkzDefPointBy[projection=onto #2--#4 ](\tkz@pttr@name \csname tkz@pt1\endcsname)
\pgfnodealias{\csname tkz@ppt2\endcsname\csname tkz@pt1\endcsname}{tkzPointResult}
\tkzDefPointBy[projection=onto #2--#4 ](\tkz@pttr@name \csname tkz@pt2\endcsname)
\pgfnodealias{\csname tkz@ppt2\endcsname\csname tkz@pt2\endcsname}{tkzPointResult}
\tkzDefPointBy[projection=onto #2--#4 ](\tkz@pttr@name \csname tkz@pt3\endcsname)
\pgfnodealias{\csname tkz@ppt2\endcsname\csname tkz@pt3\endcsname}{tkzPointResult}
\tkzDefPointBy[projection=onto #3--#2 ](\tkz@pttr@name \csname tkz@pt1\endcsname)
\pgfnodealias{\csname tkz@ppt3\endcsname\csname tkz@pt1\endcsname}{tkzPointResult}
\tkzDefPointBy[projection=onto #3--#2 ](\tkz@pttr@name \csname tkz@pt2\endcsname)
\pgfnodealias{\csname tkz@ppt3\endcsname\csname tkz@pt2\endcsname}{tkzPointResult}
\tkzDefPointBy[projection=onto #3--#2 ](\tkz@pttr@name \csname tkz@pt3\endcsname)
\pgfnodealias{\csname tkz@ppt3\endcsname\csname tkz@pt3\endcsname}{tkzPointResult}
\endgroup
}
%<--------------------------------------------------------------------------–>
%              Point on circle
%<--------------------------------------------------------------------------–>
\pgfkeys{/tkzptcircle/.cd,
          angle/.store in     = \tkz@angle,
          angle               = 0 ,
          center/.store  in   =  \tkz@center,
          radius/.store in    = \tkz@radius
}
\def\tkzDefPointOnCircle{\pgfutil@ifnextchar[{\tkz@DefPointOnCircle}{\tkz@DefPointOnCircle[]}}
\def\tkz@DefPointOnCircle[#1]{%
\begingroup
\pgfqkeys{/tkzptcircle}{#1}
\path (\tkz@center) --++(\tkz@angle:\tkz@radius) coordinate(tkzPointResult);
\endgroup
}
%<--------------------------------------------------------------------------–>
%              Point on line
%<--------------------------------------------------------------------------–>
\def\tkzDefPointOnLine{\pgfutil@ifnextchar[{\tkz@DefPointOnLine}{\tkz@DefPointOnLine[]}}
\def\tkz@DefPointOnLine[#1](#2,#3){%
\begingroup
\path (#2) to [#1] coordinate (tkzPointResult)  (#3);
\endgroup
}

\makeatother
\endinput
