% tkz-obj-eu-circles.tex
% Copyright 2020  Alain Matthes
% This work may be distributed and/or modified under the
% conditions of the LaTeX Project Public License, either version 1.3
% of this license or (at your option) any later version.
% The latest version of this license is in
%   http://www.latex-project.org/lppl.txt
% and version 1.3 or later is part of all distributions of LaTeX
% version 2005/12/01 or later.
%
% This work has the LPPL maintenance status “maintained”.
%
% The Current Maintainer of this work is Alain Matthes.

%  utf8 encoding
\def\fileversion{3.05c}
\def\filedate{2020/03/08}
\typeout{2020/03/08 3.05c  tkz-obj-eu-circles.tex}
\makeatletter
%<--------------------------------------------------------------------------–>
%            tkzCircle center and one point
%<--------------------------------------------------------------------------–>
% attention radius circle is defined by center and a point on the circle
% R defined by center and the value of the radius
% no need to define a circle with R tikz uses this method.
% through instead of radius
\def\tkz@numc{0}
\pgfkeys{/tkzcircle/.cd,
      through/.code                  = \def\tkz@numc{0},
      radius/.code                   = \def\tkz@numc{0},
      diameter/.code                 = \def\tkz@numc{1},
      circum/.code                   = \def\tkz@numc{2},
      in/.code                       = \def\tkz@numc{3},
      ex/.code                       = \def\tkz@numc{4},
      euler/.code                    = \def\tkz@numc{5},
      nine/.code                     = \def\tkz@numc{5},
      apollonius/.code               = \def\tkz@numc{6},
      orthogonal from/.code args     = {#1}{\def\tkz@ptfrom{#1}
                                            \def\tkz@numc{7}},
      orthogonal through/.code args  = {#1 and #2}{\def\tkz@ptone{#1}
                                                   \def\tkz@pttwo{#2}
                                                   \def\tkz@numc{8}},
      spieker/.code                  = \def\tkz@numc{9},
      K/.code                        = \def\tkz@koeff{#1},
      K                       =  1,
      through
      }
\def\tkzDefCircle{\pgfutil@ifnextchar[{\tkz@DefCircle}{\tkz@DefCircle[]}}
\def\tkz@DefCircle[#1](#2){%
\begingroup
\pgfqkeys{/tkzcircle}{#1}
\ifcase\tkz@numc%
  \tkzDefCircleThrough(#2)%
  \or% 1
  \tkzDefCircleD(#2)
  \or% 2
  \tkzDefCircumCircle(#2)
  \or% 3
  \tkzDefInCircle(#2)
  \or% 4
  \tkzDefExCircle(#2)
  \or% 5
  \tkzDefEulerCircle(#2)
  \or% 6
  \tkzDefApolloniusCircle(#2)
  \or% 7
  \tkzDefOrthogonalCircle(#2,\tkz@ptfrom)
  \or% 8
  \tkzDefOrthoThroughCircle(#2,\tkz@ptone,\tkz@pttwo)
  \or% 9
  \tkzDefSpiekerCircle(#2)
     \fi
\endgroup
}
%<--------------------------------------------------------------------------–>
%              Circum Circle
%<--------------------------------------------------------------------------–>
\def\tkzDefCircumCircle(#1,#2,#3){%
\begingroup
   \tkzCircumCenter(#1,#2,#3)
   \tkzCalcLength(#1,tkzPointResult)
\endgroup
}
%<--------------------------------------------------------------------------–>
%              In(scribe) Circle
%<--------------------------------------------------------------------------–>
\def\tkzDefInCircle(#1,#2,#3){%
\begingroup
  \tkzInCenter(#1,#2,#3)
  \pgfnodealias{tkz@incenter}{tkzPointResult}
  \tkzUProjection(#1,#3)(tkz@incenter)
  \pgfnodealias{tkzSecondPointResult}{tkzPointResult}
  \tkzCalcLength(tkzPointResult,tkz@incenter)
  \pgfnodealias{tkzPointResult}{tkz@incenter}
  \pgfnodealias{tkzFirstPointResult}{tkz@incenter}
\endgroup
}
%<--------------------------------------------------------------------------–>
%              Ex(scribe) Circle
%<--------------------------------------------------------------------------–>
\def\tkzDefExCircle(#1,#2,#3){%
\begingroup
  \tkzExCenter(#1,#2,#3)
  \pgfnodealias{tkz@excenter}{tkzPointResult}
  \tkzUProjection(#1,#3)(tkz@excenter)
  \pgfnodealias{tkzSecondPointResult}{tkzPointResult}
  \tkzCalcLength(tkzPointResult,tkz@excenter)% for tkzGetLength
  \pgfnodealias{tkzPointResult}{tkz@excenter}
  \pgfnodealias{tkzFirstPointResult}{tkz@excenter}
\endgroup
}
%<--------------------------------------------------------------------------–>
%              Radius Ex Circle
%<--------------------------------------------------------------------------–>
\def\tkzDefExRadius(#1,#2,#3){%
\begingroup
   \tkzExCenter(#1,#2,#3)
   \tkzUProjection(#1,#3)(tkzPointResult)
\endgroup
}
%<--------------------------------------------------------------------------–>
% The nine-point circle, also called Euler's circle or the Feuerbach circle
% best way Ma,Mb,Mc circum center  2020
%<--------------------------------------------------------------------------–>
\def\tkzDefEulerCircle(#1,#2,#3){%
\begingroup
 \tkzDefMidPoint(#1,#2) \pgfnodealias{tkz@e1}{tkzPointResult}
 \tkzDefMidPoint(#2,#3) \pgfnodealias{tkz@e2}{tkzPointResult}
 \tkzDefMidPoint(#1,#3) \pgfnodealias{tkz@e3}{tkzPointResult}
 \tkzCircumCenter(tkz@e1,tkz@e2,tkz@e3)
 \tkzCalcLength(tkzPointResult,tkz@e1)
\endgroup
}
\let\tkzDefNinePointsCircle\tkzEulerCircle%
\let\tkzFeuerBachCircle\tkzEulerCircle%

\def\tkzDefEulerRadius(#1,#2,#3){%
\begingroup
   \tkzEulerCenter(#1,#2,#3)
   \pgfnodealias{eur@pta}{tkzPointResult}
   \tkzDefMidPoint(#1,#2)
   \tkzCalcLength(eur@pta,tkzPointResult)
\endgroup
}
%<--------------------------------------------------------------------------–>
%                     Apollonius circle
%<--------------------------------------------------------------------------–>
\def\tkzDefApolloniusCircle(#1,#2){%
\begingroup
  \tkz@VecK[\tkz@koeff/(1+\tkz@koeff)](#1,#2)
  \pgfnodealias{tkzFirstPointResult}{tkzPointResult}
  \tkz@VecK[\tkz@koeff/(\tkz@koeff-1)](#1,#2)
  \pgfnodealias{tkzSecondPointResult}{tkzPointResult}
  \tkzDefMidPoint(tkzFirstPointResult,tkzSecondPointResult)
  \tkzCalcLength(tkzPointResult,tkzFirstPointResult)
\endgroup
}
%<--------------------------------------------------------------------------–>
%                     Apollonius radius
%<--------------------------------------------------------------------------–>
\pgfkeys{/tkzapor/.cd,
      K/.code    = \def\tkz@koeff{#1},% apollonius
      K          =  1
      }
\def\tkzDefApolloniusRadius{\pgfutil@ifnextchar[{%
    \tkz@DefApolloniusRadius}{\tkz@DefApolloniusRadius[]}}
\def\tkz@DefApolloniusRadius[#1](#2,#3){%
\begingroup
  \pgfqkeys{/tkzapor}{#1}
  \tkz@VecK[\tkz@koeff/(1+\tkz@koeff)](#2,#3)
  \pgfnodealias{apo@pta}{tkzPointResult}
  \tkz@VecK[\tkz@koeff/(\tkz@koeff-1)](#2,#3)
  \pgfnodealias{apo@ptb}{tkzPointResult}
  \tkzDefMidPoint(apo@pta,apo@ptb)
  \tkzCalcLength(tkzPointResult,apo@pta)
\endgroup
}
%<--------------------------------------------------------------------------–>
%                     Apollonius point
%<--------------------------------------------------------------------------–>

\pgfkeys{/tkzapop/.cd,
      K/.code    = \def\tkz@koeff{#1},% apollonius
      K          =  1
      }
\def\tkzDefApolloniusPoint{\pgfutil@ifnextchar[{\tkz@DefApolloniusPoint}{\tkz@DefApolloniusPoint[]}}
\def\tkz@DefApolloniusPoint[#1](#2,#3){%
\begingroup
   \pgfqkeys{/tkzapop}{#1}
   \tkzDefBarycentricPoint(#2=1,#3=\tkz@koeff)
\endgroup
}
%<--------------------------------------------------------------------------–>
%                     Apollonius center
%<--------------------------------------------------------------------------–>
\pgfkeys{/tkzapoc/.cd,
      K/.code   = \def\tkz@koeff{#1},% apollonius
      K         =  1
      }
\def\tkzApolloniusCenter{\pgfutil@ifnextchar[{\tkz@ApolloniusCenter}{\tkz@ApolloniusCenter[]}}
\def\tkz@ApolloniusCenter[#1](#2,#3){%
\begingroup
   \pgfqkeys{/tkzapoc}{#1}
    \tkz@VecK[\tkz@koeff/(1+\tkz@koeff)](#2,#3)
    \pgfnodealias{tkzFirstPointResult}{tkzPointResult}
    \tkz@VecK[\tkz@koeff/(\tkz@koeff-1)](#2,#3)
    \pgfnodealias{tkzSecondPointResult}{tkzPointResult}
    \tkzDefMidPoint(tkzFirstPointResult,tkzSecondPointResult)
\endgroup
}
%<--------------------------------------------------------------------------–>
\def\tkzDefOrthogonalCircle(#1,#2,#3){%
\begingroup
   \tkzTgtFromP(#1,#2)(#3)
   \tkzCalcLength[cm](#1,tkzFirstPointResult)
\endgroup
}
%<--------------------------------------------------------------------------–>
\def\tkzDefOrthoThroughCircle(#1,#2,#3,#4){%
\begingroup
  \tkzCalcLength[cm](#1,#3)\tkzGetLength{tkz@lnb}%
	\edef\tkz@lnc{\fpeval{1/\tkz@lnb}}
  \tkzVecKNorm[\tkz@lnc](#1,#3)
  \pgfnodealias{tkz@PointResult}{tkzPointResult}
  \tkzCircumCenter(tkz@PointResult,#3,#4)
  \tkzCalcLength(tkzPointResult,#3)
\endgroup
}
%<--------------------------------------------------------------------------–>
%              Spieker Circle
%<--------------------------------------------------------------------------–>
\def\tkzDefSpiekerCircle(#1,#2,#3){%
\begingroup
  \tkzSpiekerCenter(#1,#2,#3)
  \pgfnodealias{tkz@spka}{tkzPointResult}
  \tkzDefMidPoint(#1,#2)
  \tkzUProjection(#1,#2)(tkzPointResult)
  \tkzCalcLength(tkz@spka,tkzPointResult)
\endgroup
}
%<--------------------------------------------------------------------------–>
%              End Def Circle
%<--------------------------------------------------------------------------–>
 \makeatother
\endinput
